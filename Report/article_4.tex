%%%%%%%%%%%%%%%%%%%%%%%%%%%%%%%%%%%%%%%%%
% Arsclassica Article
% LaTeX Template
% Version 1.1 (1/8/17)
%
% This template has been downloaded from:
% http://www.LaTeXTemplates.com
%
% Original author:
% Lorenzo Pantieri (http://www.lorenzopantieri.net) with extensive modifications by:
% Vel (vel@latextemplates.com)
%
% License:
% CC BY-NC-SA 3.0 (http://creativecommons.org/licenses/by-nc-sa/3.0/)
%
%%%%%%%%%%%%%%%%%%%%%%%%%%%%%%%%%%%%%%%%%

%----------------------------------------------------------------------------------------
%	PACKAGES AND OTHER DOCUMENT CONFIGURATIONS
%----------------------------------------------------------------------------------------

\documentclass[
10pt, % Main document font size
a4paper, % Paper type, use 'letterpaper' for US Letter paper
oneside, % One page layout (no page indentation)
%twoside, % Two page layout (page indentation for binding and different headers)
headinclude,footinclude, % Extra spacing for the header and footer
BCOR5mm, % Binding correction
]{scrartcl}
\usepackage{float}
\input{structure.tex} % Include the structure.tex file which specified the document structure and layout

\hyphenation{Fortran hy-phen-ation} % Specify custom hyphenation points in words with dashes where you would like hyphenation to occur, or alternatively, don't put any dashes in a word to stop hyphenation altogether

%----------------------------------------------------------------------------------------
%	TITLE AND AUTHOR(S)
%----------------------------------------------------------------------------------------

\title{\normalfont\spacedallcaps{Article Title}} % The article title

%\subtitle{Subtitle} % Uncomment to display a subtitle

\author{\spacedlowsmallcaps{Fabrizio Cominetti, Davide Abete, Ruben Agazzi}} % The article author(s) - author affiliations need to be specified in the AUTHOR AFFILIATIONS block

\date{} % An optional date to appear under the author(s)

%----------------------------------------------------------------------------------------

\begin{document}

%----------------------------------------------------------------------------------------
%	HEADERS
%----------------------------------------------------------------------------------------

\renewcommand{\sectionmark}[1]{\markright{\spacedlowsmallcaps{#1}}} % The header for all pages (oneside) or for even pages (twoside)
%\renewcommand{\subsectionmark}[1]{\markright{\thesubsection~#1}} % Uncomment when using the twoside option - this modifies the header on odd pages
\lehead{\mbox{\llap{\small\thepage\kern1em\color{halfgray} \vline}\color{halfgray}\hspace{0.5em}\rightmark\hfil}} % The header style

\pagestyle{scrheadings} % Enable the headers specified in this block

%----------------------------------------------------------------------------------------
%	TABLE OF CONTENTS & LISTS OF FIGURES AND TABLES
%----------------------------------------------------------------------------------------

\maketitle % Print the title/author/date block

\setcounter{tocdepth}{2} % Set the depth of the table of contents to show sections and subsections only

\tableofcontents % Print the table of contents


%----------------------------------------------------------------------------------------
%	ABSTRACT
%----------------------------------------------------------------------------------------

\section*{Abstract} % This section will not appear in the table of contents due to the star (\section*)

Il progetto realizzato consiste nella creazione di una base di dati a grafo contenente le relazioni fra i vari prodotti Marvel. All'interno del database sono presenti relazioni come ad esempio fra personaggi e fumetti, personaggi e personaggi, etc... 

%----------------------------------------------------------------------------------------
%	AUTHOR AFFILIATIONS
%----------------------------------------------------------------------------------------


%----------------------------------------------------------------------------------------

\newpage % Start the article content on the second page, remove this if you have a longer abstract that goes onto the second page

%----------------------------------------------------------------------------------------
%	INTRODUCTION
%----------------------------------------------------------------------------------------

\section{Introduzione}
Lo scopo del progetto consiste nella realizzazione di un database a grafo riguardante vari prodotti Marvel, fra cui personaggi, film, serie tv e fumetti.
L'obiettivo del progetto è di ottenere una sorta di "rete sociale" di super eroi o personaggi marvel, i quali sono collegati ai rispettivi fumetti, film , serie tv, o rispettivi collegamenti interpersonali fra personaggi e personaggi.
 
%----------------------------------------------------------------------------------------
%	METHODS
%----------------------------------------------------------------------------------------

\section{Fonti dati}
Per l'ottenimento dei dati sono stati utilizzati due metodi diversi: l'utilizzo di una Web API e Web Scraping.

\subsection{Web Scraping}
Per la parte riguardante il Web Scraping come sorgente dati è stata utilizata la Marvel Cinematic Universe Wiki.

Lo scraping è stato effettuato eseguendo un notebook python.
I passi eseguiti dal notebook per completare lo scraping sono:
\begin{enumerate}
\item Ottenimento dalla pagina relativa a tutti i personaggi della wiki i nomi dei personaggi con i relativi link alla pagina personale.
\item per ogni personaggio aprire la pagina personale e ottenere sempre tramite scraping le informazioni rilevanti, come ad esempio: biografia, lista di film in cui è presente il personaggio, lista di serie tv in cui è presente il personaggio e relazioni con altri personaggi.
\item Per ogni film trovato nelle pagine dei personaggi aprire la pagina relativa al film e ottenere tramite scraping le informazioni del film come ad esempio la trama, i registi, scrittori, compositori, incasso, data di uscita e durata del film.
\item Per ogni serie trovata nelle pagine dei personaggi aprire la pagina relativa alla serie e ottenere tramite scraping le informazioni della serie in questione, come ad esempio la trama, i registi, i produttori e i compositori
\item Salvataggio temporaneo all'interno di file csv di film, serie tv e personaggi per essere processati in seguito.
\end{enumerate}

\subsection{Web Api}
Per l'ottenimento dei dati tramite web API è stata utilizzato il servizio formito dalla Marvel che mette a disposizione una sua Web Api per ottenere i dati relativi a personaggi e fumetti.
L'ottenimento dei dati è avvenuto tramite esecuzione di un notebook python.
L'esecuzione consiste nei seguenti passi:
\begin{enumerate}
\item Ottenimento dei dati dei personaggi in formato JSON tramite chiamata al relativo endpoint della web api a gruppi di 100 personaggi, in quanto è il limite imposto dagli sviluppatori del servizio.
\item Salvataggio su file CSV dei dati relativi ai personaggi ottenuti.
\item Ottenimento dei dati dei fumetti in formato JSON tramite chiamata al relativo endpoint della web api a gruppi di 100 fumetti, in quanto è il limite imposto dagli sviluppatori del servizio.
\item Salvataggio su file CSV dei dati relativi ai personaggi ottenuti.
\end{enumerate}
\section{Data Exploration}
La fase di data exploration è stata realizzata sempre utilizzando python con alcune librerie grafiche per la creazione di visualizzazioni.

\subsection{Data Quality}
Le principali problematiche riscontrate nella qualità dei dati sono state riscontrate nelle biografie dei personaggi e nelle relazioni tra personaggi ottenute tramite web scraping;
\subsubsection{Data Quality e Cleaning Biografia}
Per quanto riguarda la qualità dei dati delle biografie le principali problematiche sono:
\begin{enumerate}
	\item Il notebook salva i vari paragrafi delle biografie come una lista di stringhe, quindi prima di tutto la biografia viene trasformata in una stringa, concatenando le varie stringhe presenti nella lista e inserendo un carattere newLine fra un elemento e l'altro.
	\item Il notebook inoltre salvava alcuni paragrafi della biografia più volte, per questo prima di concatenare la stringa viene effettuato un controllo per vedere se la porzione di stringa è già stata aggiunta alla stringa finale
	\item Infine viene fatto un escaping dei caratteri speciali, come ad esempio il carattere " o ', in modo da non avere problemi nell'inserimento nel database dei dati.
\end{enumerate}
\subsubsection{Data Quality e Cleaning Relazioni}
Per quanto riguarda la qualità dei dati delle relazioni interpersonali ottenute tramite web scraping la principale problematica consiste nell'assenza in alcune relazioni del nome del personaggio interessato, ad esempio ci sono relazioni che hanno come nome personaggio "Mother", "Father" e così via.
Per risolvere questa problematica il notebook esegue le seguenti operazioni:
\begin{enumerate}
	\item Per ogni personaggio vengono recuperate le relative relazioni.
	\item Per ogni relazione ottenuta viene effettuato un controllo sul nome del personaggio della relazione, se il nome non è presente all'interno della lista di personaggi la relazione viene scartata
\end{enumerate}
\begin{figure}[H]
  \includegraphics[scale=0.5]{plot_relazioni.png}
  \caption{Grafico relativo alle relazioni tenute rispetto a quelle scartate}
\end{figure}
\subsection{Completezza}
Sono state effettuate altre analisi riguardanti la completezza dei dati: in particolare al riguardo di attributi dei dati ottenuti.
\subsubsection{Dati personaggi}
Le analisi di completezza effettuate sui personaggi consistono nel confronto fra i dati ottenuti dalla web api e quelli ottenuti tramite web scraping. In particolare si può riscontrare una corrispondenza di 323 personaggi della web api con quelli ottenuti tramite web scraping, al netto dei 1559 personaggi della web api.
\begin{figure}[H]
  \includegraphics[scale=0.5]{plot_corrispondenza_personaggi.png}
  \caption{Grafico al numero di personaggi che sono collegabili con i dati ottenuti tramite web scraping.}
\end{figure}
\subsubsection{Descrizione fumetti}
Le analisi di completezza effettuate sulle descrizioni dei fumetti, ottenute tramite web api, consiste nel vedere quanti fumetti possiedono una descrizione. In particolare circa 19000 fumetti non hanno descrizione al netto dei circa 50000 fumetti.
\begin{figure}[H]
  \includegraphics[scale=0.5]{plot_descrizione_fumetti.png}
  \caption{Grafico relativo al numero di fumetti con e senza descrizione}
\end{figure}
\section{Data Integration}
\subsection{Schema matching}
\section{Database}
\subsection{Struttura Database}
\subsection{Nodi Database}
\subsection{Relazioni database}
\subsection{Statistiche database}

%----------------------------------------------------------------------------------------
%	RESULTS AND DISCUSSION
%----------------------------------------------------------------------------------------


%----------------------------------------------------------------------------------------
%	BIBLIOGRAPHY
%----------------------------------------------------------------------------------------

\renewcommand{\refname}{\spacedlowsmallcaps{References}} % For modifying the bibliography heading

\bibliographystyle{unsrt}

\bibliography{sample.bib} % The file containing the bibliography

%----------------------------------------------------------------------------------------

\end{document}